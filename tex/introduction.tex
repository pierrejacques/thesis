%# -*- coding: utf-8-unix -*-
%%==================================================
%% chapter01.tex for SJTU Master Thesis
%%==================================================

%\bibliographystyle{sjtu2}%[此处用于每章都生产参考文献]
\chapter{概述}
\label{chap:introduction}
长期以来,对美学的研究一直在文化、艺术等领域展开。进化论美学和神经美学的提出,以及为其作证的东非萨凡纳(Savanna)生境【】等现象为通过科学手段研究美学提供了有力的支撑。进化论美学的观点认为审美是一种逐步进化得到的对周遭环境是否利于生存的快速判断能力【】,这意味着美与信息接收的效率息息相关。而人的视觉系统——包括眼球和与它连接的大脑皮层,作为这种视觉信息的接收和处理系统,以它特定的模式决策它对视觉信息的获取:人的视觉就如同一盏聚光灯,拥有一个狭窄而高精度的中央视觉(),环绕着一个范围宽阔而低精度的周围视觉()。视觉注意力决策这盏聚光灯的移动方向,聚焦到人脸、文字、屏幕上的图片等各类有意义的目标上去,以便从杂乱的背景中提取出有效的信息。通过不停地从一个注意点和跳转到下一个注意点,发掘零碎的有效信息,最终能在脑内形成一个对观察物的整体的印象。

流畅理论【】,在进化论美学观点的基础上揭示了人的视觉流动与美的关系——美感的反馈越是积极,则会有越是流畅的视觉流动,意味着越是轻松的注意力决策以及越是高效的视觉信息传输效率。进化论美学与流畅假设的观点的启发让我们相信视觉的复杂度和视觉重点的分布与美感存在着显著的联系。

为了验证这一猜想,我们让三十个被试分别浏览四十张在美感的好坏上具有代表性的网页,每张页面曝光3000毫秒。通过眼动仪记下的他们的视觉行为数据,结合之后他们对每个网页基于自己的审美给出的美感评价进行数据分析。通过引入信息熵来考量视觉的流畅性,分析结果表明基于用于可视化视觉重点分布的热图的信息熵与用户给出的美感评分表现出了显著的相关性($r=-0.65, F=26.84, P=0.7\times 10^{-6}$)。仅仅这一项单一指标可以对网页的美感好坏作出$87.5\%$的分类正确率。熵在这里代表一种对视觉注意过程中的混乱度的客观量化。从而证实了视觉复杂度、视觉重点与美感之间的联系。直接证实了流畅理论的观点。

那么是否通过从网页截图中提取的复杂度和视觉重点相关的特征能够对网页的美感具有足够的推测力呢?我们对1447张网页截图进行复杂度与视觉重点相关的进行特征提取,结合这些网页的线上众包美感评分的数据,通过机器学习手段进行训练得到的模型。该模型在交叉验证中取得了$83\%$的分类成功率,进一步证实和深化了视觉复杂度、视觉重点分布与美感的紧密联系。

最后,以上述的研究结果作为理论支撑,我们给出对视觉复杂度和视觉重点分布的度量在网页版式评分上的一个应用。讨论该系统的形态、应用场景、工程框架、技术实现等方面的细节,并给出可演示的原型美感评分系统。
