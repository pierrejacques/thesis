%# -*- coding: utf-8-unix -*-
%%==================================================
%% conclusion.tex for SJTUThesis
%% Encoding: UTF-8
%%==================================================

\begin{summary}

本文通过一个眼动实证研究,一个图像特征推测力研究和一个网页版式评分系统的搭建,在理论和应用层面多角度地证实了进化论美学和流畅理论关于视觉复杂度和视觉注意力与美感的关系的猜想。眼动实验提出了视觉注意熵的概念,论证了拥有较强美感的审美对象会导致较小的相对视觉注意熵;图像特征提取实验通过对网格复杂度、占空分布、信息密度分布、视觉显著性分布等特征的提取及验证、论证了网页截图图像中关于视觉复杂度的信息与关于视觉重点分布的特征对美感的显著推测能力;最后,对美感评分系统的具体的工程实现和83\%的网页版式好坏的区分正确率,切实给出了机器获得审美的可行性和技术架构。

上述结论表明,从理论层面上,审美体验遵循最小代价最大收益的原则\citen{Hekkert2006}。“美即是可用”\citen{Tractinsky2000}的观点至少对于先天性的审美而言是正确的。

工程上,网页版式评分系统本身还有诸多值得探索和突破的细节。一个全面而强大的美感评分系统应该是由多个评估模型(如版式、色彩、物件识别、语义等)整合而成的,能对审美对象进行全方位评价的系统。

设计自动化亦是美感评分系统的一个重要的应用方向。基于高效的版式生成式系统和优秀的美感评分系统的生成式设计系统的诞生是令人期待的。然而毋庸置疑,对美学的深入理解和研究是获得高效且优雅\footnote{此处“优雅”指结构清晰可解释}的基石。

\end{summary}
