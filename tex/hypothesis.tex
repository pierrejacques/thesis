\chapter{关于审美主体和客体的理论假设}
\label{chap:hypothesis}
我们的设想源自流畅理论\cite{Reber2004, Reber2012}。浏览一个网页的过程本质上是人类的视觉系统处理一张图像的过程。如何来评估这种视觉处理的流畅性呢?我们可以通过视觉处理管道的入口,神经和认知系统的上游——视觉注意,来入手。根本上,视觉注意可以定义为从全部可达的信息中选出一个用于进一步处理的子集的过程。视觉注意持续地通过自底向上(画面视觉重点的驱动)和自定向下(诱人的内容的驱动)的方式在外围视力中选择目标。审美主体的视觉注意受他对审美对象的视觉复杂度(visual complexity)和视觉重点分布(即显著性分布,visual saliency)两个部分。前者决定了耗费视觉注意资源量的大前提,后者则可以在相同的视觉复杂度下降低或增高视觉处理的代价。

视觉接收通道可以理解成一个具有带宽(视觉信息接收能力)上限的网络信息传输通道。如果流畅理论成立的话,则注意过程将是美感评价的途径。具体而言,一个能够造成正面审美反馈的对象的视觉复杂度应该在一个不过高亦不过低的合理的区间范围内,在此基础上的视觉重点分布应该能够恰当地引导视线,从而提高视觉信息传输的效率。

合理而流畅的视觉注意应满足:
\begin{enumerate}
  \item 在面对多个视觉注意线索时,更少冲突地作出选择.
  \item 在面对局部视觉重点时,更聚焦地趋向一个兴趣点
\end{enumerate}

本研究中,对于审美主体,我们引入信息熵的概念来量化视觉注意行为的混乱度(不流畅性),并探究它是否跟美感评价相关。熵将被应用在眼动数据的以下两个方面:

\begin{enumerate}
  \item 眼动注意的转移序列的熵:反映注意力在多个视觉重点间转移的复杂度和不确定性
  \item 眼动注意的热图熵:反映眼动注意在空间分布上的分散度和混乱度。
\end{enumerate}

对于审美客体,我们引入一些已经被提出或是新的用于度量视觉复杂度和视觉显著性分布的图像特征,并探究由他们搭建的学习模型对美感的推测力。一个具有较高美感评价的页面的视觉复杂度和视觉重点分布应该有以下的特点:

\begin{enumerate}
  \item 视觉复杂度适中,不过高也不过低
  \item 视觉重点的空间分布具有适度的不平衡性,不过于平均也不过于失衡
\end{enumerate}
