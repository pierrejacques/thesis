\chapter{相关研究}
\label{chap:related}

\section{美学理论研究}
美学(Aesthetics)一词的词源来自希腊语,表示“观察”。纵观人类的历史,人类不论其文明背景、生活方式和发展程度,都对美有着不懈的追求。对美感的定义至今存在着诸多的争议,原因在于很多定义是建立在主观感知的假设上的,例如“人类对美的感知的心理和情绪”假设所有人都对什么是美(Beauty)有着一致而明确的认识。这样的定义的过于主观,无法解释人与人之间的差别。有一些定义方式是基于典型范例的,比如定义“红色”为鲜血、成熟的草莓的颜色。然而这样的定义方式对美学不适用,因为这样的话要定义美学为“看到梵高的星夜、米开朗基罗的大卫等名作时的心理感受”,这样的定义一方面存在着个体差异过大的问题。另一方面,美感,或者说审美反应(Aesthetic Response)并非只对美的事物发生,对于几乎所有的被视觉感知到的对象发生【Palmer et al 2012, Reber et al 2012】。这也是美感与艺术审美的一大区别,艺术审美只对特定的人造对象发生。【】大部分时候,审美过程是一念之间的,无处不在地发生在人类的潜意识中。但在某些情况下,例如产生强烈的审美反馈或是抱有明确审美目的的时候,审美行为会进入意识层面。【】

美的哲学讨论开始于柏拉图与拉里士多德。近代康德(Kant)对心理美学的观点是颇有影响力的,他把美学认为是一种观者的心理体验而不是一种客体的物理属性。【】“美学判断包含了三个关键特征:主观、无功利和普适性”。【Dickie, 1997】无功利性要求美学的探讨不应包含欲望,例如相对小的蛋糕更喜欢大的蛋糕是因为食欲。而康德认为美学包含着相比仅仅的认同和个人的喜好而言更为复杂的认知。他把这种复杂性表述为“和谐的自由的想象”(free play of the imagination)。【】现今一般认同康德的主观和普适的观点而对无功利性的持保留态度,后续的一些研究和现象证实,美感最初恰恰是为了生存这样的功利性目的而进化产生的【】。

美学是否能通过现代科学手段进行研究呢?一些学者认为答案是肯定的【e.g., Arnheim 1974,Berlyne 1971, Fechner 1876, Jacobson 2006,Shimamura \& Palmer 2012】, 一些学者则抱有消极的态度【Dickie 1962】,这些学者认为由于科学的客观性和条款性,用科学手段研究美学是不可行且自相矛盾的。纵然,审美毫无疑问是主观的,但这并不意味着他们不具有被客观研究的可能性。举例而言,人类对颜色的认知是主观的,但是仍有许多已经建立的较为完备的色彩科学体系【e.g Kaiser \& Boynton 1996, Koenderink 2010】。美学的科学化研究并不是去判定一个事物或者一个图片是否是客观上美的,而是去判断一个代表集的个体是否会认为他是美的。从而美学的科学包含了精确地描述人们的美学判断,以及探索这种判断产生的原因。

随着现代科学的发展,客观地定义审美成为可能。在研究方法上,以神经实验手段研究美学的领域被称为神经美学(Neuroaesthetics)【5、6、31】。该领域的学者认为,美感可以定义为大脑特定的区域的神经活跃【Ramachandan \& Hirstein、Zeki】。而在对美的产生机理上,与神经美学有着紧密关联的进化论美学【4】的进展使我们对人的视觉美感过程有了更深入的认识。这些新的美学研究区别于传统哲学美学和应用美学,更注重通过科学手段进行理论假设和实证检验。

进化美学认为,人类的审美是逐步发展起来的,审美起源自人类因生存需要而进化出的对周遭环境的一种视觉本能和感性知识的积淀【4】。审美不是完全先天而稳定客观的,也不是完全后天而差异主观的。审美的先天部分,具有跨越种族的一致性和稳定性,包含了对生存最为必要的本能判断。而后天养成的审美与我们的经验系统有关,在很大程度上同样有助于我们更好的适应后天的生存环境。后天审美受知识、记忆、环境等诸多因素影响而在不同文化、不同群体间表现出显著的差异,这是造成很多人认为审美没有统一标准的原因。一些早期的美学计算实验如O’Donovan【11】、Reinecke【18】等没有有意识地区分源自进化的美感和后天习得的经验性美感。

对于进化论美学的观点,有诸多的自然现象作为依据。其中较为著名的是东非稀树草原的萨凡纳生境(Savanna)现象。大量的例证资料

对于美学,一个自然的研究思路是“视觉行为能一定程度上反映我们对视觉刺激物的感受”。视觉定位由视觉系统——眼球和与之相连的大脑皮层神经系统,控制。如果把视觉行为理解为一系列具有信息获取能力的注视行为和对下一个注视位置进行决策的扫视行为的话。视觉注意力就是驱动这些注视和扫视行为的决策性动力。流畅理论【】在进化论美学的基本观点上对视觉行为与美感之间的联系提出猜想:越是能造成正面美学体验的刺激物,对他的视觉过程应该越是流畅的【】。这种流畅性在视觉注意力上表现为较低的决策负担,意味着视觉能在更短的时间内把更多的有用信息从背景中剥离出来,以及花费更少的能量进行决策。这样的猜想与进化论美学的核心观点是一致的。

\section{眼动与美学实证研究}

在考察眼动与对艺术品的美感评价之间的联系方面,Berlyne【】认为美感评价是基于两种视觉行为的:一种是整体而多样的探索性扫掠,一种是局部而具体的,以信息获取为目的的聚焦。前者具有较广的视线范围和较短暂的注视时长,后者具有较小的视线范围和较长的注视时长。Berlyne提出这种由注视时长和扫视范围交替构成的眼动探索模式对于图片的美感满意度的评价是至关重要的,该探索性视觉的理念进一步影响了如下的一系列研究。Locher【】发现简单地对一个抽象组合对象的色彩平衡进行调整会造成眼动注视分布和视觉路劲的改变;Franke et al【】发现更受好评的三维渲染图像往往有更多的眼动注视个数和更长的注视时长;Plumhoff【】发现对于好的图像,眼动注视的时长更长,眼动扫视的范围随时间表现出更大的变化;Wallraven et al【】分析了20名被试对275个不同时期的艺术作品的眼动数据,发现不同风格的作品的眼动注视个数和时长之间存在较强的差异;Massaro et al【】对美术作品进行归类(彩色的、灰度的、人文的、自然的),并以此作为研究视觉注意力中自底向上和自定向下过程的贡献的实验材料;Khalighy et al【】通过三组基于抽象图像和产品设计的眼动实验推导出一个关于美感的经验公式,他认为美感与注视的个数和注视时长之间的方差的乘积呈正比。

上述的研究解释了眼动追踪技术对于多种形式对象的美感研究的潜在可能性,但是在本质上,他们的成果没有超出Berlyne的想法。类似诸如更多的注视个数、更长的注视时长、随时间更富变化的扫视范围等实验结果,仅仅加强证实了Berlyne的观点——对于具有更高美感评价的对象会获得更活跃而动态的眼动行为反馈。事实上,这些结果很难得到合理的美学角度的解释。尽管眼动仪已经成为美学研究的新装备,但到目前为止,就眼动行为是如何与美感反馈产生联系的仍然缺乏深入和令人信服的解释。


\section{眼动与网页美感}

网页的美感是吸引用户的重要因素:大数据研究表明绝大多数的用户在进入一个网页的初始几秒内就决定去留【】。而好看的网页带给用户更强的可信感【】,看起来更好用【】。

网页美感的研究主要都聚焦在从网页中提取具有美感推测力的特征上(例如复杂度【】),并取得了一定的成果。Ivory和Sinha【】、Zhang et al.【】、Reinecke et al.【】通过低级的图像特征来预测感知视觉复杂度和彩度,并发现它们与人们的美感评价高度相关。

眼动仪被广泛用来评估对网页上特定元素的视觉注意度,然而现有的眼动数据的可视化和分析手段与美感相关的用户体验没有任何关联,关于眼动和美感的关系仍有待证实【】。眼动仪能否提供一个更为普遍的度量观者的美感体验的手段?我们需要通过美学的视角借助一些数据挖掘的手段,来从眼动数据中提取出更多的可解释的指标。


\section{网页版式评分系统}

网页的版式评分系统往往被应用在自动化生成式设计系统中。一些版式评分系统采用基于规则的专家模式,如Gaudii平面设计专家系统【10】,利用专家设定的128条模糊逻辑规则作为适应函数,在对海报的版式美感评价中取得了较好的美学效果。但是,现实中的设计存在大量原则之外的特例,手工定义所有版式规则和适用条件是几乎不可能的。因此,基于统计学习的路径更值得期待,2014年多伦多大学O’Donovan开发了一个带有学习能力的版式系统 【11】,采用了一个由122个版式特征变量的能量函数评估自动生成的方案,该函数中各项的权重设置以类似模拟退火的非线性逆优化(NIO)方式从设计师选择的案例中学习,取得了一定的效果。不足的是该函数的本质是案例模仿,而不是更具推广性的审美计算模型,且这些版式特征量的有效性并没有经过大样本量的验证和筛选。总体而言,网页版式评价在数据采集、特征选择和学习方法上还有很大的尝试和发展空间。
