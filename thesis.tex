%# -*- coding: utf-8-unix -*-
%%==================================================
%% thesis.tex
%%==================================================

% 双面打印
\documentclass[doctor, fontset=mac, openright, twoside]{sjtuthesis}
% \documentclass[bachelor, fontset=adobe, openany, oneside, submit]{sjtuthesis}
% \documentclass[master, fontset=adobe, review]{sjtuthesis}
% \documentclass[%
%   bachelor|master|doctor,	% 必选项
%   fontset=adobe|windows,  	% 只测试了adobe
%   oneside|twoside,		% 单面打印,双面打印(奇偶页交换页边距,默认)
%   openany|openright, 		% 可以在奇数或者偶数页开新章|只在奇数页开新章(默认)
%   zihao=-4|5,, 		% 正文字号:小四、五号(默认)
%   review,	 		% 盲审论文,隐去作者姓名、学号、导师姓名、致谢、发表论文和参与的项目
%   submit			% 定稿提交的论文,插入签名扫描版的原创性声明、授权声明
% ]

% 逐个导入参考文献数据库
\addbibresource{bib/thesis.bib}
% \addbibresource{bib/chap2.bib}

\begin{document}

%% 无编号内容:中英文论文封面、授权页
%# -*- coding: utf-8-unix -*-
\title{基于进化论美学的网页版式评分系统}
\author{金辰浩}
\advisor{顾振宇教授}
% \coadvisor{某某教授}
\defenddate{2017年11月10日}
\school{上海交通大学}
\institute{媒体与设计学院}
\studentnumber{115200910075}
\major{工业设计工程}

\englishtitle{Webpage Layout Evaluating System based on Evolutional Aesthetics}
\englishauthor{\textsc{Chenhao Jin}}
\englishadvisor{Prof. \textsc{Zhenyu Gu}}
% \englishcoadvisor{Prof. \textsc{Uom Uom}}
\englishschool{Shanghai Jiao Tong University}
\englishinstitute{\textsc{School of Media and Design} \\
  \textsc{Shanghai Jiao Tong University} \\
  \textsc{Shanghai, P.R.China}}
\englishmajor{Industrial Design Engineering}
\englishdate{Nov. 10th, 2017}

\maketitle

\makeenglishtitle

\makeatletter
\ifsjtu@submit\relax
	\includepdf{pdf/original.pdf}
	\cleardoublepage
	\includepdf{pdf/authorization.pdf}
	\cleardoublepage
\else
\ifsjtu@review\relax
% exclude the original claim and authorization
\else
	\makeDeclareOriginal
	\makeDeclareAuthorization
\fi
\fi
\makeatother


\frontmatter 	% 使用罗马数字对前言编号

%% 摘要
\pagestyle{main}
%# -*- coding: utf-8-unix -*-
%%==================================================
%% abstract.tex for SJTU Master Thesis
%%==================================================

\begin{abstract}

进化论美学的提出为美学使用科学方法进行研究提供了可行性,让通过心理学、统计学、信息论等手段探索美的本质成为可能。

基于进化论美学的流畅理论认为视觉流畅性与美感之间存在正相关。本文使用眼动仪对30个被试者对40张网页页面的最初3秒的眼动行为的进行记录,引入香农信息熵对数据进行分析。结果表明基于热图的眼动熵(视觉注意熵)与被试者对网格美感的评价存在着显著的相关性。其改进版本——相对视觉注意熵有更高的相关性(相关系数$r = -0.65$,显著性$P < 0.0001$)。此单指标对好坏美感的样本的线性分类准确率就达到了$87.5\%$。进一步的分析表明视觉注意熵的表现在曝光开始1秒后达到稳定。发展曲线表明如果实验时间超过3秒,它的表现甚至可以更好。

在证实视觉行为与美感的关系后,本文论证从图片中提取的视觉复杂度与视觉显著性相关的特征对美感的推测能力。通过对收集1447张网页截图进行众包评分和版式特征提取,发现网格复杂度、信息密度、占空和显著性都分别对美感评分表现除了较高的组间区分度。

最后本文从工程上给出上述研究的一个应用——网页版式评分系统。基于有效特征构建的评分系统取得了对样本页面$83\%$的平均交叉验证分类准确率。

\keywords{进化论美学 \quad 流畅理论 \quad 视觉显著性 \quad 网页版式 \quad 特征提取}
\end{abstract}

\begin{englishabstract}

Evolutionary aesthetics made it possible to study aesthetics using scientific methods introduced from psychology, statistics and information theory.

Fluency theory based on evolutionary aesthetics proposed that observer's aesthetics response is positively correlated to the fluency of his/her eye movement. This study tracked 30 observers' initial landings for 40 web pages (each displayed for 3 seconds) and introduced Shannon Entropy to analyze the data. The result shows that the heatmap entropy (visual attention entropy) is highly correlated with the observers' aesthetic judgements of the web pages. Its improved version, the relative Visual Attention Entropy, has a more significant correlation with the perceived aesthetics($r = -0.65, P < 0.0001$). Theis single metric along can distinguish between good- and bad-looking pages with an accuracy of $87.5\%$. Further investigating reveals that the performance of Visual Attention Entropy became stable after 1 second of exposure. The outcome indicated that the performance could be better if the tracking time was extended beyond 3 seconds.

After the prove of the correlation between eye movement fluency and perceived aesthetics, this study dicusses graphic features' predictive ability to aesthetic ratings.
We collected a total amount of 1447 web pages and rated them by conducting an online survey. By extracting graphic features that are related to visual complexity and visual saliency, we found grid complexity, margin distribution, information density and saliency distribution performed significant variance between good- and bad-looking groups.

Finally we realized an application for the aforementioned results, a web layout evaluating system, and reached a $83\%$ average cross validation accuracy. 

\englishkeywords{evolutional aesthetics, fluency theory, visual saliency, webpage layout, feature extracting}
\end{englishabstract}


%% 目录、插图目录、表格目录
\tableofcontents
\listoffigures
\addcontentsline{toc}{chapter}{\listfigurename} %将插图目录加入全文目录
\listoftables
\addcontentsline{toc}{chapter}{\listtablename}  %将表格目录加入全文目录
\listofalgorithms
\addcontentsline{toc}{chapter}{算法索引}        %将算法目录加入全文目录

\include{tex/symbol} % 主要符号、缩略词对照表

\mainmatter	% 使用阿拉伯数字对正文编号

%% 正文内容
\pagestyle{main}
%# -*- coding: utf-8-unix -*-
%%==================================================
%% chapter01.tex for SJTU Master Thesis
%%==================================================

%\bibliographystyle{sjtu2}%[此处用于每章都生产参考文献]
\chapter{概述}
\label{chap:intro}

这是上海交通大学(非官方)学位论文 \LaTeX 模板,当前版本是 \version 。

最早的一版学位模板是一位热心的物理系同学制作的。
那份模板参考了自动化所学位论文模板,使用了CASthesis.cls文档类,中文字符处理则采用当时最为流行的 \CJKLaTeX 方案。
我根据交大研究生院对学位论文的要求
\footnote{\url{http://www.gs.sjtu.edu.cn/policy/fileShow.ahtml?id=130}}
,结合少量个人审美喜好,完成了一份基本可用的交大 \LaTeX 学位论文模板。
但是,搭建一个 \CJKLaTeX 环境并不简单,单单在Linux下配置环境和添加中文字体,就足够让新手打退堂鼓。
在William Wang的建议下,我开始着手把模板向 \XeTeX 引擎移植。
他完成了最初的移植,多亏了他出色的工作,后续的改善工作也得以顺利进行。

随着我对 \LaTeX 系统认知增加,我又断断续续做了一些完善模板的工作,在原有硕士学位论文模板的基础上完成了交大学士和博士学位论文模板。

现在,交大学位论文模板SJTUTHesis代码在github
\footnote{\url{https://github.com/weijianwen/SJTUThesis}}
上维护。
你可以\href{https://github.com/weijianwen/SJTUThesis/issues}{在github上开issue}
、或者在\href{https://bbs.sjtu.edu.cn/bbsdoc?board=TeX_LaTeX}{水源LaTeX版}发帖来反映遇到的问题。

\section{使用模板}

\subsection{准备工作}
\label{sec:requirements}

要使用这个模板撰写学位论文,需要在\emph{TeX系统}、\emph{中英文字体}、\emph{TeX技能}上有所准备。

\begin{itemize}[noitemsep,topsep=0pt,parsep=0pt,partopsep=0pt]
	\item {\TeX}系统:所使用的{\TeX}系统要支持 \XeTeX 引擎,且带有ctex 2.x宏包,以2015年的\emph{完整}TeXLive、MacTeX发行版为佳。
	\item 中英文字体:操作系统中需要安装\footnote{在Windows、Mac OS X 以及 Linux 上安装额外的字体,可以参考\href{https://www.searchfreefonts.com/articles/how-to-install-fonts.htm}{“How to install fonts?”}。
}TeX Gyre Termes字体\footnote{\url{http://www.gust.org.pl/projects/e-foundry/tex-gyre/termes}}和四款Adobe中文字体
\footnote{请从合法渠道获得Adobe字体。}:AdobeSongStd、AdobeKaitiStd、AdobeHeitiStd、AdobeFangsongStd。
	\item TeX技能:尽管提供了对模板的必要说明,但这不是一份“ \LaTeX 入门文档”。在使用前请先通读其他入门文档。
	\item 针对Windows用户的额外需求:学位论文模本分别使用git和GNUMake进行版本控制和构建,建议从Cygwin\footnote{\url{http://cygwin.com}}安装这两个工具。
\end{itemize}

\subsection{模板选项}
\label{sec:thesisoption}

sjtuthesis提供了一些常用选项,在thesis.tex在导入sjtuthesis模板类时,可以组合使用。
这些选项包括:

\begin{itemize}[noitemsep,topsep=0pt,parsep=0pt,partopsep=0pt]
\item 学位类型:bachelor(学位)、master(硕士)、doctor(博士),是必选项。
\item 中国字体:adobefonts(Adobe中文字体)、winfonts(使用Windows下的中文字体,该选项未在Linux/Mac下测试)。
\item 正文字号:cs4size(小四)、c5size(五号)。
\item 盲审选项:使用review选项后,论文作者、学号、导师姓名、致谢、发表论文和参与项目将被隐去。
\end{itemize}

\subsection{编译模板}
\label{sec:process}

模板默认使用GNUMake构建,GNUMake将调用latemk工具自动完成模板多轮编译:

\begin{lstlisting}[basicstyle=\small\ttfamily, caption={编译模板}, numbers=none]
make clean thesis.pdf
\end{lstlisting}

若需要生成包含“原创性声明扫描件”的学位论文文档,请将扫描件保存为statement.pdf,然后调用make生成submit.pdf。

\begin{lstlisting}[basicstyle=\small\ttfamily, caption={生成用于提交的学位论文}, numbers=none]
make clean submit.pdf
\end{lstlisting}

编译失败时,可以尝试手动逐次编译,定位故障。

\begin{lstlisting}[basicstyle=\small\ttfamily, caption={手动逐次编译}, numbers=none]
xelatex -no-pdf thesis
biber --debug thesis
xelatex thesis
xelatex thesis
\end{lstlisting}

\subsection{模板文件布局}
\label{sec:layout}

\begin{lstlisting}[basicstyle=\small\ttfamily,caption={模板文件布局},label=layout,float,numbers=none]
├── LICENSE
├── Makefile
├── README.md
├── bib
│   ├── chap1.bib
│   └── chap2.bib
├── bst
│   └── GBT7714-2005NLang.bst
├── figure
│   ├── chap2
│   │   ├── sjtulogo.eps
│   │   ├── sjtulogo.jpg
│   │   ├── sjtulogo.pdf
│   │   └── sjtulogo.png
│   └── sjtubanner.png
├── sjtuthesis.cfg
├── sjtuthesis.cls
├── statement.pdf
├── submit.pdf
├── tex
│   ├── abstract.tex
│   ├── ack.tex
│   ├── app_cjk.tex
│   ├── app_eq.tex
│   ├── app_log.tex
│   ├── chapter01.tex
│   ├── chapter02.tex
│   ├── chapter03.tex
│   ├── conclusion.tex
│   ├── id.tex
│   ├── patents.tex
│   ├── projects.tex
│   ├── pub.tex
│   └── symbol.tex
└── thesis.tex
\end{lstlisting}

本节介绍学位论文模板中木要文件和目录的功能。

\subsubsection{格式控制文件}
\label{sec:format}

格式控制文件控制着论文的表现形式,包括以下几个文件:
sjtuthesis.cfg, sjtuthesis.cls和GBT7714-2005NLang.bst。
其中,“cfg”和“cls”控制论文主体格式,“bst”控制参考文献条目的格式,

\subsubsection{主控文件thesis.tex}
\label{sec:thesistex}

主控文件thesis.tex的作用就是将你分散在多个文件中的内容“整合”成一篇完整的论文。
使用这个模板撰写学位论文时,你的学位论文内容和素材会被“拆散”到各个文件中:
譬如各章正文、各个附录、各章参考文献等等。
在thesis.tex中通过“include”命令将论文的各个部分包含进来,从而形成一篇结构完成的论文。
对模板定制时引入的宏包,建议放在导言区。

\subsubsection{各章源文件tex}
\label{sec:thesisbody}

这一部分是论文的主体,是以“章”为单位划分的,包括:

\begin{itemize}[noitemsep,topsep=0pt,parsep=0pt,partopsep=0pt]
	\item 中英文摘要(abstract.tex)。前言(frontmatter)的其他部分,中英文封面、原创性声明、授权信息在sjtuthesis.cls中定义,不单独分离为tex文件。
不单独弄成文件。
	\item 正文(mainmatter)——学位论文正文的各章内容,源文件是chapter\emph{xxx}.tex。
	\item 附录(app\emph{xx}.tex)、致谢(thuanks.tex)、攻读学位论文期间发表的学术论文目录(pub.tex)、个人简历(resume.tex)组成正文后的部分(backmatter)。
参考文献列表由bibtex插入,不作为一个单独的文件。
\end{itemize}

\subsubsection{图片文件夹figure}
\label{sec:fig}

figure文件夹放置了需要插入文档中的图片文件(支持PNG/JPG/PDF/EPS格式的图片),可以在按照章节划分子目录。
模板文件中使用\verb|\graphicspath|命令定义了图片存储的顶层目录,在插入图片时,顶层目录名“figure”可省略。

\subsubsection{参考文献数据库bib}
\label{sec:bib}

目前参考文件数据库目录只存放一个参考文件数据库thesis.bib。
关于参考文献引用,可参考第\ref{chap:example}章中的例子。

%# -*- coding: utf-8-unix -*-
%%==================================================
%% chapter02.tex for SJTU Master Thesis
%% based on CASthesis
%% modified by wei.jianwen@gmail.com
%% Encoding: UTF-8
%%==================================================

\chapter{ \LaTeX 排版例子}
\label{chap:example}

\section{列表环境}
\label{sec:list}

\subsection{无序列表}
\label{sec:unorderlist}

以下是一个无序列表的例子,列表的每个条目单独分段。

\begin{itemize}
  \item 这是一个无序列表。
  \item 这是一个无序列表。
  \item 这是一个无序列表。
\end{itemize}

使用\verb+itemize*+环境可以创建行内无序列表。
\begin{itemize*}
  \item 这是一个无序列表。
  \item 这是一个无序列表。
  \item 这是一个无序列表。
\end{itemize*}
行内无序列表条目不单独分段,所有内容直接插入在原文的段落中。

\subsection{有序列表}
\label{sec:orderlist}

使用环境\verb+enumerate+和\verb+enumerate*+创建有序列表,
使用方法无序列表类似。

\begin{enumerate}
  \item 这是一个有序列表。
  \item 这是一个有序列表。
  \item 这是一个有序列表。
\end{enumerate}

使用\verb+enumerate*+环境可以创建行内有序列表。
\begin{enumerate*}
  \item 这是一个默认有序列表。
  \item 这是一个默认有序列表。
  \item 这是一个默认有序列表。
\end{enumerate*}
行内有序列表条目不单独分段,所有内容直接插入在原文的段落中。

\subsection{描述型列表}

使用环境\verb+description+可创建带有主题词的列表,条目语法是\verb+\item[主题] 内容+。
\begin{description}
    \item[主题一] 详细内容
    \item[主题二] 详细内容
    \item[主题三] 详细内容 \ldots
\end{description}

\subsection{自定义列表样式}

可以使用\verb+label+参数控制列表的样式,
详细可以参考WikiBooks\footnote{\url{https://en.wikibooks.org/wiki/LaTeX/List_Structures\#Customizing_lists}}。
比如一个自定义样式的行内有序列表
\begin{enumerate*}[label=\itshape\alph*)\upshape]
  \item 这是一个自定义样式有序列表。
  \item 这是一个自定义样式有序列表。
  \item 这是一个自定义样式有序列表。
\end{enumerate*}

\section{数学排版}
\label{sec:matheq}

\subsection{公式排版}
\label{sec:eqformat}

这里有举一个长公式排版的例子,来自\href{http://www.tex.ac.uk/tex-archive/info/math/voss/mathmode/Mathmode.pdf}{《Math mode》}:

\begin {multline}
  \frac {1}{2}\Delta (f_{ij}f^{ij})=
  2\left (\sum _{i<j}\chi _{ij}(\sigma _{i}-
    \sigma _{j}) ^{2}+ f^{ij}\nabla _{j}\nabla _{i}(\Delta f)+\right .\\
  \left .+\nabla _{k}f_{ij}\nabla ^{k}f^{ij}+
    f^{ij}f^{k}\left [2\nabla _{i}R_{jk}-
      \nabla _{k}R_{ij}\right ]\vphantom {\sum _{i<j}}\right )
\end{multline}

\subsection{SI单位}

使用\verb+siunitx+宏包可以方便地输入SI单位制单位,例如\verb+\SI{5}{\um}+可以得到\SI{5}{\um}。

\subsubsection{一个四级标题}
\label{sec:depth4}

这是全文唯一的一个四级标题。在这部分中将演示了mathtools宏包中可伸长符号(箭头、等号的例子)的例子。

\begin{displaymath}
    A \xleftarrow[n=0]{} B \xrightarrow[LongLongLongLong]{n>0} C
\end{displaymath}

\begin{eqnarray}
  f(x) & \xleftrightarrow[]{A=B}  & B \\
  & \xleftharpoondown[below]{above} & B \nonumber \\
  & \xLeftrightarrow[below]{above} & B
\end{eqnarray}

又如:

\begin{align}
  \label{eq:none}
  & I(X_3;X_4)-I(X_3;X_4\mid{}X_1)-I(X_3;X_4\mid{}X_2) \nonumber \\
  = & [I(X_3;X_4)-I(X_3;X_4\mid{}X_1)]-I(X_3;X_4\mid{}\tilde{X}_2) \\
  = & I(X_1;X_3;X_4)-I(X_3;X_4\mid{}\tilde{X}_2)
\end{align}

\subsection{定理环境}

模板中定义了丰富的定理环境
algo(算法),thm(定理),lem(引理),prop(命题),cor(推论),defn(定义),conj(猜想),exmp(例),rem(注),case(情形),
bthm(断言定理),blem(断言引理),bprop(断言命题),bcor(断言推论)。
amsmath还提供了一个proof(证明)的环境。
这里举一个“定理”和“证明”的例子。
\begin{thm}[留数定理]
\label{thm:res}
  假设$U$是复平面上的一个单连通开子集,$a_1,\ldots,a_n$是复平面上有限个点,$f$是定义在$U\backslash \{a_1,\ldots,a_n\}$上的全纯函数,
  如果$\gamma$是一条把$a_1,\ldots,a_n$包围起来的可求长曲线,但不经过任何一个$a_k$,并且其起点与终点重合,那么:

  \begin{equation}
    \label{eq:res}
    \ointop_{\gamma}f(z)\,\mathrm{d}z = 2\uppi\mathbf{i}\sum^n_{k=1}\mathrm{I}(\gamma,a_k)\mathrm{Res}(f,a_k)
  \end{equation}

  如果$\gamma$是若尔当曲线,那么$\mathrm{I}(\gamma, a_k)=1$,因此:

  \begin{equation}
    \label{eq:resthm}
    \ointop_{\gamma}f(z)\,\mathrm{d}z = 2\uppi\mathbf{i}\sum^n_{k=1}\mathrm{Res}(f,a_k)
  \end{equation}

      % \oint_\gamma f(z)\, dz = 2\pi i \sum_{k=1}^n \mathrm{Res}(f, a_k ).

  在这里,$\mathrm{Res}(f, a_k)$表示$f$在点$a_k$的留数,$\mathrm{I}(\gamma,a_k)$表示$\gamma$关于点$a_k$的卷绕数。
  卷绕数是一个整数,它描述了曲线$\gamma$绕过点$a_k$的次数。如果$\gamma$依逆时针方向绕着$a_k$移动,卷绕数就是一个正数,
  如果$\gamma$根本不绕过$a_k$,卷绕数就是零。

  定理\ref{thm:res}的证明。

  \begin{proof}
    首先,由……

    其次,……

    所以……
  \end{proof}
\end{thm}

上面的公式例子中,有一些细节希望大家注意。微分号d应该使用“直立体”也就是用mathrm包围起来。
并且,微分号和被积函数之间应该有一段小间隔,可以插入\verb+\,+得到。
斜体的$d$通常只作为一般变量。
i,j作为虚数单位时,也应该使用“直立体”为了明显,还加上了粗体,例如\verb+\mathbf{i}+。斜体$i,j$通常用作表示“序号”。
其他字母在表示常量时,也推荐使用“直立体”譬如,圆周率$\uppi$(需要upgreek宏包),自然对数的底$\mathrm{e}$。
不过,我个人觉得斜体的$e$和$\pi$很潇洒,在不至于引起混淆的情况下,我也用这两个字母的斜体表示对应的常量。


\section{向文档中插入图像}
\label{sec:insertimage}

\subsection{支持的图片格式}
\label{sec:imageformat}

\XeTeX 可以很方便地插入PDF、PNG、JPG格式的图片。

插入PNG/JPG的例子如\ref{fig:SRR}所示。
这两个水平并列放置的图共享一个“图标题”(table caption),没有各自的小标题。

\begin{figure}[!htp]
  \centering
  \includegraphics[width=0.3\textwidth]{example/sjtulogo.png}
  \hspace{1cm}
  \includegraphics[width=0.3\textwidth]{example/sjtulogo.jpg}
  \bicaption[fig:SRR]{这里将出现在插图索引中}{中文题图}{Fig}{English caption}
\end{figure}

% 这里还有插入eps图像和pdf图像的例子,如图\ref{fig:epspdf:a}和图\ref{fig:epspdf:b}。这里将EPS和PDF图片作为子图插入,每个子图有自己的小标题。并列子图的功能是使用subfigure宏包提供的。
%
% \begin{figure}
%   \centering
%   \subfigure[EPS Figure]{
%     \label{fig:epspdf:a} %% label for first subfigure
%     \includegraphics[width=0.3\textwidth]{example/sjtulogo.eps}}
%   \hspace{1in}
%   \subfigure[PDF Figure]{
%     \label{fig:epspdf:b} %% label for second subfigure
%     \includegraphics[width=0.3\textwidth]{example/sjtulogo.pdf}}
%   \bicaption[fig:pdfeps]{插入eps图像和pdf图像}{插入eps和pdf的例子}{Fig}{An EPS and PDF demo}
% \end{figure}

更多关于 \LaTeX 插图的例子可以参考\href{http://www.cs.duke.edu/junhu/Graphics3.pdf}{《\LaTeX 插图指南》}。

\subsection{长标题的换行}
\label{sec:longcaption}

图\ref{fig:longcaptionbad}和图\ref{fig:longcaptiongood}都有比较长图标题,通过对比发现,图\ref{fig:longcaptiongood}的换行效果更好一些。
其中使用了minipage环境来限制整个浮动体的宽度。

\begin{figure}[!htp]
 \centering
 \includegraphics[width=4cm]{example/sjtulogo.pdf}
 \bicaption[fig:longcaptionbad]{这里将出现在插图索引}{海交通大学是我国历史最悠久的高等学府之一,是教育部直属、教育部与上海市共建的全国重点大学.}{Fig}{Where there is a will, there is a way.}
\end{figure}

\begin{figure}[!htbp]
  \centering
  \begin{minipage}[b]{0.6\textwidth}
    \captionstyle{\centering}
    \centering
    \includegraphics[width=4cm]{example/sjtulogo.pdf}
    \bicaption[fig:longcaptiongood]{这里将出现在插图索引}{海交通大学是我国历史最悠久的高等学府之一,是教育部直属、教育部与上海市共建的全国重点大学.}{Fig}{Where there is a will, there is a way.}
  \end{minipage}
\end{figure}

\subsection{绘制流程图}

图\ref{fig:flow_chart}是一张流程图示意。使用tikz环境,搭配四种预定义节点(\verb+startstop+、\verb+process+、\verb+decision+和\verb+io+),可以容易地绘制出流程图。
\begin{figure}[!htp]
    \centering
    \resizebox{6cm}{!}{\input{figure/example/flow_chart.tex}}
    \bicaption[fig:flow_chart]{绘制流程图效果}{流程图}{Fig}{Flow chart}
\end{figure}

\clearpage

\section{表格}
\label{sec:tab}

这一节给出的是一些表格的例子,如表\ref{tab:firstone}所示。

\begin{table}[!hpb]
  \centering
  \bicaption[tab:firstone]{指向一个表格的表目录索引}{一个颇为标准的三线表格\footnotemark[1]}{Table}{A Table}
  \begin{tabular}{@{}llr@{}} \toprule
    \multicolumn{2}{c}{Item} \\ \cmidrule(r){1-2}
    Animal & Description & Price (\$)\\ \midrule
    Gnat & per gram & 13.65 \\
    & each & 0.01 \\
    Gnu & stuffed & 92.50 \\
    Emu & stuffed & 33.33 \\
    Armadillo & frozen & 8.99 \\ \bottomrule
  \end{tabular}
\end{table}
\footnotetext[1]{这个例子来自\href{http://www.ctan.org/tex-archive/macros/latex/contrib/booktabs/booktabs.pdf}{《Publication quality tables in LATEX》}(booktabs宏包的文档)。这也是一个在表格中使用脚注的例子,请留意与threeparttable实现的效果有何不同。}

下面一个是一个更复杂的表格,用threeparttable实现带有脚注的表格,如表\ref{tab:footnote}。

\begin{table}[!htpb]
  \bicaption[tab:footnote]{出现在表目录的标题}{一个带有脚注的表格的例子}{Table}{A Table with footnotes}
  \centering
  \begin{threeparttable}[b]
     \begin{tabular}{ccd{4}cccc}
      \toprule
      \multirow{2}{6mm}{total}&\multicolumn{2}{c}{20\tnote{1}} & \multicolumn{2}{c}{40} &  \multicolumn{2}{c}{60}\\
      \cmidrule(lr){2-3}\cmidrule(lr){4-5}\cmidrule(lr){6-7}
      &www & k & www & k & www & k \\
      \midrule
      &$\underset{(2.12)}{4.22}$ & 120.0140\tnote{2} & 333.15 & 0.0411 & 444.99 & 0.1387 \\
      &168.6123 & 10.86 & 255.37 & 0.0353 & 376.14 & 0.1058 \\
      &6.761    & 0.007 & 235.37 & 0.0267 & 348.66 & 0.1010 \\
      \bottomrule
    \end{tabular}
    \begin{tablenotes}
    \item [1] the first note.% or \item [a]
    \item [2] the second note.% or \item [b]
    \end{tablenotes}
  \end{threeparttable}
\end{table}

\section{参考文献管理}

 \LaTeX 具有将参考文献内容和表现形式分开管理的能力,涉及三个要素:参考文献数据库、参考文献引用格式、在正文中引用参考文献。
这样的流程需要多次编译:

\begin{enumerate}[noitemsep,topsep=0pt,parsep=0pt,partopsep=0pt]
	\item 用户将论文中需要引用的参考文献条目,录入纯文本数据库文件(bib文件)。
	\item 调用xelatex对论文模板做第一次编译,扫描文中引用的参考文献,生成参考文献入口文件(aux)文件。
	\item 调用bibtex,以参考文献格式和入口文件为输入,生成格式化以后的参考文献条目文件(bib)。
	\item 再次调用xelatex编译模板,将格式化以后的参考文献条目插入正文。
\end{enumerate}

参考文献数据库(thesis.bib)的条目,可以从Google Scholar搜索引擎\footnote{\url{https://scholar.google.com}}、CiteSeerX搜索引擎\footnote{\url{http://citeseerx.ist.psu.edu}}中查找,文献管理软件Papers\footnote{\url{http://papersapp.com}}、Mendeley\footnote{\url{http://www.mendeley.com}}、JabRef\footnote{\url{http://jabref.sourceforge.net}}也能够输出条目信息。

下面是在Google Scholar上搜索到的一条文献信息,格式是纯文本:

\begin{lstlisting}[caption={从Google Scholar找到的参考文献条目}, label=googlescholar, escapeinside="", numbers=none]
    @phdthesis{"白2008信用风险传染模型和信用衍生品的定价",
      title={"信用风险传染模型和信用衍生品的定价"},
      author={"白云芬"},
      year={2008},
      school={"上海交通大学"}
    }
\end{lstlisting}

推荐修改后在bib文件中的内容为:

\begin{lstlisting}[caption={修改后的参考文献条目}, label=itemok, escapeinside="", numbers=none]
  @phdthesis{bai2008,
    title={"信用风险传染模型和信用衍生品的定价"},
    author={"白云芬"},
    date={2008},
    address={"上海"},
    school={"上海交通大学"}
  }
\end{lstlisting}

按照教务处的要求,参考文献外观应符合国标GBT7714的要求\footnote{\url{http://www.cces.net.cn/guild/sites/tmxb/Files/19798_2.pdf}}。
在模板中,表现形式的控制逻辑通过bibla­tex-gb7714-2015包实现\footnote{\url{https://www.ctan.org/pkg/biblatex-gb7714-2015}},基于{Bib\LaTeX}管理文献。在目前的多数TeX发行版中,可能都没有默认包含biblatex-gb7714-2015,需要手动安装。

正文中引用参考文献时,用\verb+\cite{key1,key2,key3...}+可以产生“上标引用的参考文献”,
如\cite{Meta_CN,chen2007act,DPMG}。
使用\verb+\citen{key1,key2,key3...}+则可以产生水平引用的参考文献,例如\citen{JohnD,zhubajie,IEEE-1363}。
请看下面的例子,将会穿插使用水平的和上标的参考文献:关于书的\citen{Meta_CN,JohnD,IEEE-1363},关于期刊的\cite{chen2007act,chen2007ewi},
会议论文\citen{DPMG,kocher99,cnproceed},
硕士学位论文\citen{zhubajie,metamori2004},博士学位论文\cite{shaheshang,FistSystem01,bai2008},标准文件\citen{IEEE-1363},技术报告\cite{NPB2},电子文献\citen{xiaoyu2001, CHRISTINE1998},用户手册\citen{RManual}。

总结一些注意事项:
\begin{itemize}
\item 参考文献只有在正文中被引用了,才会在最后的参考文献列表中出现;
\item 参考文献“数据库文件”bib是纯文本文件,请使用UTF-8编码,不要使用GBK编码;
\item 参考文献条目中默认通过date域输入时间。兼容使用year域时会产生编译warning,可忽略。
\end{itemize}

\section{用listings插入源代码}

原先ctexbook文档类和listings宏包配合使用时,代码在换页时会出现莫名其妙的错误,后来经高人指点,顺利解决了。
感兴趣的话,可以看看\href{http://bbs.ctex.org/viewthread.php?tid=53451}{这里}。
这里给使用listings宏包插入源代码的例子,这里是一段C代码。
另外,listings宏包真可谓博大精深,可以实现各种复杂、漂亮的效果,想要进一步学习的同学,可以参考
\href{http://mirror.ctan.org/macros/latex/contrib/listings/listings.pdf}{listings宏包手册}。

\begin{lstlisting}[language={C}, caption={一段C源代码}]
#include <stdio.h>
#include <unistd.h>
#include <sys/types.h>
#include <sys/wait.h>

int main() {
  pid_t pid;

  switch ((pid = fork())) {
  case -1:
    printf("fork failed\n");
    break;
  case 0:
    /* child calls exec */
    execl("/bin/ls", "ls", "-l", (char*)0);
    printf("execl failed\n");
    break;
  default:
    /* parent uses wait to suspend execution until child finishes */
    wait((int*)0);
    printf("is completed\n");
    break;
  }

  return 0;
}
\end{lstlisting}

\section{用algorithm和algorithmicx宏包插入算法描述}

algorithmicx 比 algorithmic 增加了一些命令。
示例如算法\ref{algo:sum_100}和算法\ref{algo:merge_sort},
后者的代码来自\href{http://hustsxh.is-programmer.com/posts/38801.html}{xhSong的博客}。
algorithmicx的详细使用方法见\href{http://mirror.hust.edu.cn/CTAN/macros/latex/contrib/algorithmicx/algorithmicx.pdf}{官方README}。
使用算法宏包时,算法出现的位置很多时候不按照tex文件里的书写顺序,
需要强制定位时可以使用\verb+\begin{algorithm}[H]+
\footnote{http://tex.stackexchange.com/questions/165021/fixing-the-location-of-the-appearance-in-algorithmicx-environment}

这是写在算法\ref{algo:sum_100}前面的一段话,在生成的文件里它会出现在算法\ref{algo:sum_100}前面。

\begin{algorithm}
% \begin{algorithm}[H] % 强制定位
\caption{求100以内的整数和}
\label{algo:sum_100}
\begin{algorithmic}[1] %每行显示行号
\Ensure 100以内的整数和 % 输出
\State $sum \gets 0$
\For{$i = 0 \to 100$}
    \State $sum \gets sum + i$
  \EndFor
\end{algorithmic}
\end{algorithm}

这是写在两个算法中间的一段话,当算法\ref{algo:sum_100}不使用\verb+\begin{algorithm}[H]+时它也会出现在算法\ref{algo:sum_100}前面。

对于很长的算法,单一的算法块\verb+\begin{algorithm}...\end{algorithm}+是不能自动跨页的
\footnote{http://tex.stackexchange.com/questions/70733/latex-algorithm-not-display-under-correct-section},
会出现的情况有:

\begin{itemize}
  \item 该页放不下当前的算法,留下大片空白,算法在下一页显示
  \item 单一页面放不下当前的算法,显示时超过页码的位置直到超出整个页面范围
\end{itemize}

解决方法有:

\begin{itemize}
  \item (推荐)使用\verb+algstore{algname}+和\verb+algrestore{algname}+来讲算法分为两个部分\footnote{http://tex.stackexchange.com/questions/29816/algorithm-over-2-pages},如算法\ref{algo:merge_sort}。
  \item 人工拆分算法为多个小的部分。
\end{itemize}

\begin{algorithm}
% \begin{algorithm}[H] % 强制定位
\caption{用归并排序求逆序数}
\label{algo:merge_sort}
\begin{algorithmic}[1] %每行显示行号
\Require $Array$数组,$n$数组大小 % 输入
\Ensure 逆序数 % 输出
\Function {MergerSort}{$Array, left, right$}
  \State $result \gets 0$
  \If {$left < right$}
    \State $middle \gets (left + right) / 2$
    \State $result \gets result +$ \Call{MergerSort}{$Array, left, middle$}
    \State $result \gets result +$ \Call{MergerSort}{$Array, middle, right$}
    \State $result \gets result +$ \Call{Merger}{$Array,left,middle,right$}
  \EndIf
  \State \Return{$result$}
\EndFunction
\State %空一行
\Function{Merger}{$Array, left, middle, right$}
  \State $i\gets left$
  \State $j\gets middle$
  \State $k\gets 0$
  \State $result \gets 0$
  \While{$i<middle$ \textbf{and} $j<right$}
    \If{$Array[i]<Array[j]$}
      \State $B[k++]\gets Array[i++]$
    \Else
      \State $B[k++] \gets Array[j++]$
      \State $result \gets result + (middle - i)$
    \EndIf
  \EndWhile
  \algstore{MergeSort}
\end{algorithmic}
\end{algorithm}

\begin{algorithm}
\begin{algorithmic}[1]
  \algrestore{MergeSort}
  \While{$i<middle$}
    \State $B[k++] \gets Array[i++]$
  \EndWhile
  \While{$j<right$}
    \State $B[k++] \gets Array[j++]$
  \EndWhile
  \For{$i = 0 \to k-1$}
    \State $Array[left + i] \gets B[i]$
  \EndFor
  \State \Return{$result$}
\EndFunction
\end{algorithmic}
\end{algorithm}

这是写在算法\ref{algo:merge_sort}后面的一段话,
但是当算法\ref{algo:merge_sort}不使用\verb+\begin{algorithm}[H]+时它会出现在算法\ref{algo:merge_sort}
甚至算法\ref{algo:sum_100}前面。

对于算法的索引要注意\verb+\caption+和\verb+\label+的位置,
必须是先\verb+\caption+再\verb+\label+\footnote{http://tex.stackexchange.com/questions/65993/algorithm-numbering},
否则会出现\verb+\ref{algo:sum_100}+生成的编号跟对应算法上显示不一致的问题。

根据Werner的回答\footnote{http://tex.stackexchange.com/questions/53357/switch-cases-in-algorithmic}
增加了\verb+Switch+和\verb+Case+的支持,见算法\ref{algo:switch_example}。

\begin{algorithm}
\caption{Switch示例}
\label{algo:switch_example}
\begin{algorithmic}[1]
  \Switch{$s$}
    \Case{$a$}
      \Assert{0}
    \EndCase
    \Case{$b$}
      \Assert{1}
    \EndCase
    \Default
      \Assert{2}
    \EndDefault
  \EndSwitch
\end{algorithmic}
\end{algorithm}

\include{tex/faq}
%# -*- coding: utf-8-unix -*-
%%==================================================
%% conclusion.tex for SJTUThesis
%% Encoding: UTF-8
%%==================================================

\begin{summary}

本文通过一个眼动实证研究,一个图像特征推测力研究和一个网页版式评分系统的搭建,在理论和应用层面多角度地证实了进化论美学和流畅理论关于视觉复杂度和视觉注意力与美感的关系的猜想。眼动实验提出了视觉注意熵的概念,论证了拥有较强美感的审美对象会导致较小的相对视觉注意熵;图像特征提取实验通过对网格复杂度、占空分布、信息密度分布、视觉显著性分布等特征的提取及验证、论证了网页截图图像中关于视觉复杂度的信息与关于视觉重点分布的特征对美感的显著推测能力;最后,对美感评分系统的具体的工程实现和83\%的网页版式好坏的区分正确率,切实给出了机器获得审美的可行性和技术架构。

上述结论表明,从理论层面上,审美体验遵循最小代价最大收益的原则\citen{Hekkert2006}。“美即是可用”\citen{Tractinsky2000}的观点至少对于先天性的审美而言是正确的。

工程上,网页版式评分系统本身还有诸多值得探索和突破的细节。一个全面而强大的美感评分系统应该是由多个评估模型(如版式、色彩、物件识别、语义等)整合而成的,能对审美对象进行全方位评价的系统。

设计自动化亦是美感评分系统的一个重要的应用方向。基于高效的版式生成式系统和优秀的美感评分系统的生成式设计系统的诞生是令人期待的。然而毋庸置疑,对美学的深入理解和研究是获得高效且优雅\footnote{此处“优雅”指结构清晰可解释}的基石。

\end{summary}


\appendix	% 使用英文字母对附录编号,重新定义附录中的公式、图图表编号样式
\renewcommand\theequation{\Alph{chapter}--\arabic{equation}}
\renewcommand\thefigure{\Alph{chapter}--\arabic{figure}}
\renewcommand\thetable{\Alph{chapter}--\arabic{table}}
\renewcommand\thealgorithm{\Alph{chapter}--\arabic{algorithm}}

%% 附录内容,本科学位论文可以用翻译的文献替代。
\include{tex/app_setup}
\include{tex/app_eq}
\include{tex/app_cjk}
\include{tex/app_log}

\backmatter	% 文后无编号部分

%% 参考资料
\printbibliography[heading=bibintoc]

%% 致谢、发表论文、申请专利、参与项目、简历
%% 用于盲审的论文需隐去致谢、发表论文、申请专利、参与的项目
\makeatletter

%%
% "研究生学位论文送盲审印刷格式的统一要求"
% http://www.gs.sjtu.edu.cn/inform/3/2015/20151120_123928_738.htm

% 盲审删去删去致谢页
\ifsjtu@review\relax\else
  %# -*- coding: utf-8-unix -*-
\begin{thanks}

本文的研究,尤其是眼动实验的相关研究,历时较长。实验期间顾振宇教授长期鼓励和指导,令我有动力去探索更多的解释性指标,并最终取得一定成果。论文撰写上,顾振宇教授的反复修改与指导使结果更为深入和有说服力,令我受益匪浅。

另在实验和工程实现方面,感谢媒设2012级研究生娄坚的特征提取平台;感谢电院2017级研究生邱丰对众包实验平台的搭建上的贡献;感谢电院2017级研究生邓瀚铭对美感评分的机器学习卷积神经网络的搭建;感谢媒设2017级研究生杨秀凡对眼动实验中例外页面的设计改进;感谢媒设2015级研究生张杰琳对代码的重构意见;感谢媒设2015级研究生王靖纯对实验的协助。

感谢所有参与眼动实验和网页众包评分实验的约80名同学和师长。

\end{thanks}
 	  %% 致谢
\fi

\ifsjtu@bachelor
  % 学士学位论文要求在最后有一个英文大摘要,单独编页码
  \pagestyle{biglast}
  \include{tex/end_english_abstract}
\else
  % 盲审论文中,发表学术论文及参与科研情况等仅以第几作者注明即可,不要出现作者或他人姓名
  \ifsjtu@review\relax
    %# -*- coding: utf-8-unix -*-

\begin{publications}{99}
    \item\textsc{第一作者}. {中文非核心期刊论文}, 2017.
\end{publications}

    \include{tex/projectsreview}
  \else
    %# -*- coding: utf-8-unix -*-
%%==================================================
%% pub.tex for SJTUThesis
%% Encoding: UTF-8
%%==================================================

\begin{publications}{99}
  \item\textsc{金辰浩}. {基于互联网大数据的设计语义模型}[J]. 工业设计, 2017/10(135): 54-55.
  % \item\textsc{Chen H, Wu B~I, Zhang B}, et al. {Electromagnetic Wave Interactions with a Metamaterial Cloak}[J]. Physical Review Letters, 2007, 99(6):63903.
\end{publications}
	      %% 发表论文
    \include{tex/projects}  %% 参与的项目
  \fi
\fi

% \include{tex/patents}	  %% 申请专利
% \include{tex/resume}	  %% 个人简历

\makeatother

\end{document}
