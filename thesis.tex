%# -*- coding: utf-8-unix -*-
%%==================================================
%% thesis.tex
%%==================================================

% 双面打印
\documentclass[doctor, fontset=mac, openright, twoside]{sjtuthesis}
% \documentclass[bachelor, fontset=adobe, openany, oneside, submit]{sjtuthesis}
% \documentclass[master, fontset=adobe, review]{sjtuthesis}
% \documentclass[%
%   bachelor|master|doctor,	% 必选项
%   fontset=adobe|windows,  	% 只测试了adobe
%   oneside|twoside,		% 单面打印,双面打印(奇偶页交换页边距,默认)
%   openany|openright, 		% 可以在奇数或者偶数页开新章|只在奇数页开新章(默认)
%   zihao=-4|5,, 		% 正文字号:小四、五号(默认)
%   review,	 		% 盲审论文,隐去作者姓名、学号、导师姓名、致谢、发表论文和参与的项目
%   submit			% 定稿提交的论文,插入签名扫描版的原创性声明、授权声明
% ]

% 逐个导入参考文献数据库
\addbibresource{bib/thesis.bib}
% \addbibresource{bib/chap2.bib}

\begin{document}

%% 无编号内容:中英文论文封面、授权页
%# -*- coding: utf-8-unix -*-
\title{基于进化论美学的网页版式评分系统}
\author{金辰浩}
\advisor{顾振宇教授}
% \coadvisor{某某教授}
\defenddate{2017年11月10日}
\school{上海交通大学}
\institute{媒体与设计学院}
\studentnumber{115200910075}
\major{工业设计工程}

\englishtitle{Webpage Layout Evaluating System based on Evolutional Aesthetics}
\englishauthor{\textsc{Chenhao Jin}}
\englishadvisor{Prof. \textsc{Zhenyu Gu}}
% \englishcoadvisor{Prof. \textsc{Uom Uom}}
\englishschool{Shanghai Jiao Tong University}
\englishinstitute{\textsc{School of Media and Design} \\
  \textsc{Shanghai Jiao Tong University} \\
  \textsc{Shanghai, P.R.China}}
\englishmajor{Industrial Design Engineering}
\englishdate{Nov. 10th, 2017}

\maketitle

\makeenglishtitle

\makeatletter
\ifsjtu@submit\relax
	\includepdf{pdf/original.pdf}
	\cleardoublepage
	\includepdf{pdf/authorization.pdf}
	\cleardoublepage
\else
\ifsjtu@review\relax
% exclude the original claim and authorization
\else
	\makeDeclareOriginal
	\makeDeclareAuthorization
\fi
\fi
\makeatother


\frontmatter 	% 使用罗马数字对前言编号

%% 摘要
\pagestyle{main}
%# -*- coding: utf-8-unix -*-
%%==================================================
%% abstract.tex for SJTU Master Thesis
%%==================================================

\begin{abstract}

进化论美学的提出为美学使用科学方法进行研究提供了可行性,让通过心理学、统计学、信息论等手段探索美的本质成为可能。

基于进化论美学的流畅理论认为视觉流畅性与美感之间存在正相关。本文使用眼动仪对30个被试者对40张网页页面的最初3秒的眼动行为的进行记录,引入香农信息熵对数据进行分析。结果表明基于热图的眼动熵(视觉注意熵)与被试者对网格美感的评价存在着显著的相关性。其改进版本——相对视觉注意熵有更高的相关性(相关系数$r = -0.65$,显著性$P < 0.0001$)。此单指标对好坏美感的样本的线性分类准确率就达到了$87.5\%$。进一步的分析表明视觉注意熵的表现在曝光开始1秒后达到稳定。发展曲线表明如果实验时间超过3秒,它的表现甚至可以更好。

在证实视觉行为与美感的关系后,本文论证从图片中提取的视觉复杂度与视觉显著性相关的特征对美感的推测能力。通过对收集1447张网页截图进行众包评分和版式特征提取,发现网格复杂度、信息密度、占空和显著性都分别对美感评分表现除了较高的组间区分度。

最后本文从工程上给出上述研究的一个应用——网页版式评分系统。基于有效特征构建的评分系统取得了对样本页面$83\%$的平均交叉验证分类准确率。

\keywords{进化论美学 \quad 流畅理论 \quad 视觉显著性 \quad 网页版式 \quad 特征提取}
\end{abstract}

\begin{englishabstract}

Evolutionary aesthetics made it possible to study aesthetics using scientific methods introduced from psychology, statistics and information theory.

Fluency theory based on evolutionary aesthetics proposed that observer's aesthetics response is positively correlated to the fluency of his/her eye movement. This study tracked 30 observers' initial landings for 40 web pages (each displayed for 3 seconds) and introduced Shannon Entropy to analyze the data. The result shows that the heatmap entropy (visual attention entropy) is highly correlated with the observers' aesthetic judgements of the web pages. Its improved version, the relative Visual Attention Entropy, has a more significant correlation with the perceived aesthetics($r = -0.65, P < 0.0001$). Theis single metric along can distinguish between good- and bad-looking pages with an accuracy of $87.5\%$. Further investigating reveals that the performance of Visual Attention Entropy became stable after 1 second of exposure. The outcome indicated that the performance could be better if the tracking time was extended beyond 3 seconds.

After the prove of the correlation between eye movement fluency and perceived aesthetics, this study dicusses graphic features' predictive ability to aesthetic ratings.
We collected a total amount of 1447 web pages and rated them by conducting an online survey. By extracting graphic features that are related to visual complexity and visual saliency, we found grid complexity, margin distribution, information density and saliency distribution performed significant variance between good- and bad-looking groups.

Finally we realized an application for the aforementioned results, a web layout evaluating system, and reached a $83\%$ average cross validation accuracy. 

\englishkeywords{evolutional aesthetics, fluency theory, visual saliency, webpage layout, feature extracting}
\end{englishabstract}


%% 目录、插图目录、表格目录
\tableofcontents
\listoffigures
\addcontentsline{toc}{chapter}{\listfigurename} %将插图目录加入全文目录
\listoftables
\addcontentsline{toc}{chapter}{\listtablename}  %将表格目录加入全文目录
\listofalgorithms
\addcontentsline{toc}{chapter}{算法索引}        %将算法目录加入全文目录

\include{tex/symbol} % 主要符号、缩略词对照表

\mainmatter	% 使用阿拉伯数字对正文编号

%% 正文内容
\pagestyle{main}
%# -*- coding: utf-8-unix -*-
%%==================================================
%% chapter01.tex for SJTU Master Thesis
%%==================================================

%\bibliographystyle{sjtu2}%[此处用于每章都生产参考文献]
\chapter{概述}
\label{chap:introduction}
长期以来,对美学的研究一直在文化、艺术等领域展开。进化论美学和神经美学的提出,以及为其作证的东非萨凡纳(Savanna)生境【】等现象为通过科学手段研究美学提供了有力的支撑。进化论美学的观点认为审美是一种逐步进化得到的对周遭环境是否利于生存的快速判断能力【】,这意味着美与信息接收的效率息息相关。而人的视觉系统——包括眼球和与它连接的大脑皮层,作为这种视觉信息的接收和处理系统,以它特定的模式决策它对视觉信息的获取:人的视觉就如同一盏聚光灯,拥有一个狭窄而高精度的中央视觉(),环绕着一个范围宽阔而低精度的周围视觉()。视觉注意力决策这盏聚光灯的移动方向,聚焦到人脸、文字、屏幕上的图片等各类有意义的目标上去,以便从杂乱的背景中提取出有效的信息。通过不停地从一个注意点和跳转到下一个注意点,发掘零碎的有效信息,最终能在脑内形成一个对观察物的整体的印象。

流畅理论【】,在进化论美学观点的基础上揭示了人的视觉流动与美的关系——美感的反馈越是积极,则会有越是流畅的视觉流动,意味着越是轻松的注意力决策以及越是高效的视觉信息传输效率。进化论美学与流畅假设的观点的启发让我们相信视觉的复杂度和视觉重点的分布与美感存在着显著的联系。

为了验证这一猜想,我们让三十个被试分别浏览四十张在美感的好坏上具有代表性的网页,每张页面曝光3000毫秒。通过眼动仪记下的他们的视觉行为数据,结合之后他们对每个网页基于自己的审美给出的美感评价进行数据分析。通过引入信息熵来考量视觉的流畅性,分析结果表明基于用于可视化视觉重点分布的热图的信息熵与用户给出的美感评分表现出了显著的相关性($r=-0.65, F=26.84, P=0.7e-6$)。仅仅这一项单一指标可以对网页的美感好坏作出$87.5\%$的分类正确率。熵在这里代表一种对视觉注意过程中的混乱度的客观量化。从而证实了视觉复杂度、视觉重点与美感之间的联系。直接证实了流畅理论的观点。

那么是否通过从网页截图中提取的复杂度和视觉重点相关的特征能够对网页的美感具有足够的推测力呢?我们对1447张网页截图进行复杂度与视觉重点相关的进行特征提取,结合这些网页的线上众包美感评分的数据,通过机器学习手段进行训练得到的模型。该模型在交叉验证中取得了$83\%$的分类成功率,进一步证实和深化了视觉复杂度、视觉重点分布与美感的紧密联系。

最后,以上述的研究结果作为理论支撑,我们给出对视觉复杂度和视觉重点分布的度量在网页版式评分上的一个应用。讨论该系统的形态、应用场景、工程框架、技术实现等方面的细节,并给出可演示的原型美感评分系统。

\chapter{眼动、网页美感与版式特征提取的相关研究}
\label{chap:related}

\section{美学的定义和哲学研究}
美学(Aesthetics)一词的词源来自希腊语$\alpha\iota\sigma\theta\eta\tau\iota{\kappa}o\zeta$,表示“观察”。对于美感的定义至今存在着诸多的争议,原因在于很多定义是建立在主观感知的假设上的。一些定义的过于主观,无法解释人与人之间的差别。例如“人类对美的感知的心理和情绪”的定义假设所有人都对什么是美(Beauty)有着一致而明确的认识。有一些定义方式是基于典型范例的,类似于定义“红色”为鲜血、成熟的草莓的颜色。然而这样的定义方式对美学不适用,因为这样的话要定义美学为“看到梵高的星夜、米开朗基罗的大卫等名作时的心理感受”。这样的定义一方面同样存在着个体差异过大的问题,另一方面,美感,或者说审美反应并非只对“美”的事物发生,而对于几乎所有被视觉感知到的对象都会发生\citen{Palmer2012b, Reber2012}。这也是美感与艺术审美的一大区别,艺术审美只对特定的人造对象发生。大部分时候,审美过程是一念之间地发生在人类的潜意识之中的。但在某些情况下,例如产生强烈的审美反馈或是抱有明确审美目的的时候,审美行为会进入意识层面。

美的哲学讨论开始于柏拉图与亚里士多德。近代康德(Kant)对心理美学的观点是颇有影响力的,他把美学认为是一种观者的心理体验而不是一种客体的物理属性\citen{kant1892}。美学判断包含了三个关键特征:主观、无功利和普适性\citen{dickie1997}。无功利性要求美学的探讨不应包含欲望,例如相对小的蛋糕更喜欢大的蛋糕是因为食欲的功利性。而康德认为美学包含着相比仅仅的认同和个人的喜好而言更为复杂的认知。他把这种复杂性表述为“和谐的自由的想象”\footnote{英文原句为:"free play of the imagination"}。现今一般认同康德的主观和普适的观点而对无功利性的持保留态度。

\section{美学的科学研究与进化论美学}
美学是否能通过现代科学手段进行研究呢? 很多学者给出了肯定的答案和相关的理论\citen{Arnheim1974, Berlyne1971, Fechner1876, Jacobsen2006, Shimamura2012}, 也有学者则抱有消极的态度\citen{Dickie1962}。持消极态度的学者认为由于科学的客观性和条款性,用科学手段研究美学是不可行且自相矛盾的。纵然,审美毫无疑问是主观的,但这并不意味着他们不具有被客观研究的可能性。举例而言,人类对颜色的认知是主观的,但是仍有许多已经建立的较为完备的色彩科学体系\citen{Kaiser1996, Koenderink2010}。美学的科学化研究并不是去判定一个事物或者一个图片是否是客观上美的,而是去判断一个代表集的个体是否会认为他是美的。从而美学的科学包含了精确地描述人们的美学判断,以及探索这种判断产生的原因。

随着现代科学的发展使客观地定义审美成为可能。在研究方法上,以神经实验手段研究美学的领域被称为神经美学(Neuroaesthetics)\citen{Cinzia2009, Jacobs2003, Cela2011}。该领域的学者认为,美感可以定义为大脑特定的区域的神经活跃\citen{Ramachandran1999}。而在对美的产生机理上,与神经美学有着紧密关联的进化论美学\citen{Stoddart1997}的进展使我们对人的审美过程有了更深入的认识。这些新的美学研究区别于传统哲学美学和应用美学,更注重通过科学手段进行理论假设和实证检验。

进化论美学认为,人类的审美是逐步发展起来的,审美起源自人类因生存需要而进化出的对周遭环境的一种视觉本能和感性知识的积淀\citen{Stoddart1997}。审美不是完全先天而稳定客观的,也不是完全后天而差异主观的。审美的先天部分,具有跨越种族的一致性和稳定性,包含了对生存最为必要的本能判断。而后天养成的审美与我们的经验系统有关,在很大程度上同样有助于我们更好的适应后天的生存环境。后天审美受知识、记忆、环境等诸多因素影响而在不同文化、不同群体间表现出显著的差异,这是造成很多人认为审美没有统一标准的原因。

进化论美学的观点源于东非稀树草原的萨凡纳生境(Savanna)现象\citen{Ruso2003}:研究表明,最近的显著塑造了人类的生物适应性的生存环境是更新世\footnote{更新世:Pleistocene}的东非萨凡纳生境。它是一种稀树草原的混合生境,具有适度的生物复杂度和秩序性。进化心理学家认为即使到现代,人类的婴儿仍是生而为了长成100,000年前他们的祖先那样的采猎者的。因而我们的感知、识别、情绪、动机和行为的适应性更接近我们的祖先\citen{Miller1994}。所以人类的视觉感知偏好从某种意义上更接近萨凡纳生境的状态,即适当和视觉复杂度和适当的视觉秩序性。

对于美学,一个自然的研究思路是“视觉行为能一定程度上反映我们对视觉刺激物的感受”。视觉定位由视觉系统——眼球和与之相连的大脑皮层神经系统,控制。如果把视觉行为理解为一系列具有信息获取能力的注视行为和对下一个注视位置进行决策的扫视行为的话。视觉注意力就是驱动这些注视和扫视行为的决策性动力。流畅理论\citen{Reber2004, Reber2012}在进化论美学的基本观点上对视觉行为与美感之间的联系提出猜想:越是能造成正面美学体验的刺激物,对他的视觉过程应该越是流畅的。这种流畅性在视觉注意力上表现为较低的决策负担,意味着视觉能在更短的时间内把更多的有用信息从背景中剥离出来,以及花费更少的能量进行决策。这样的猜想与进化论美学的核心观点是一致的。

\section{眼动与美学实证研究}

在考察眼动与对艺术品的美感评价之间的联系方面,Berlyne 1971 认为美感评价是基于两种视觉行为的:一种是整体而多样的探索性扫掠,一种是局部而具体的,以信息获取为目的的聚焦\citen{Berlyne1971}。前者具有较广的视线范围和较短暂的注视时长,后者具有较小的视线范围和较长的注视时长。Berlyne提出这种由注视时长和扫视范围交替构成的眼动探索模式对于图片的美感满意度的评价是至关重要的,该探索性视觉的理念进一步影响了如下的一系列研究。Locher \& Paul 2006发现简单地对一个抽象组合对象的色彩平衡进行调整会造成眼动注视分布和视觉路劲的改变\citen{Locher2006};Franke et al. 2008 发现更受好评的三维渲染图像往往有更多的眼动注视个数和更长的注视时长\citen{Franke2008};Plumhoff发现对于好的图像,眼动注视的时长更长,眼动扫视的范围随时间表现出更大的变化\citen{Plumhoff2009};Wallraven et al.分析了20名被试对275个不同时期的艺术作品的眼动数据,发现不同风格的作品的眼动注视个数和时长之间存在较强的差异\citen{Wallraven2009};Massaro et al.对美术作品进行归类(彩色的、灰度的、人文的、自然的),并以此作为研究视觉注意力中自底向上和自顶向下过程的贡献的实验材料\citen{Massaro};Khalighy et al.通过三组基于抽象图像和产品设计的眼动实验推导出一个关于美感的经验公式,他认为美感与注视的个数和注视时长之间的方差的乘积呈正比\citen{Khalighy2015}。

上述的研究解释了眼动追踪技术对于多种形式对象的美感研究的潜在可能性,但是在本质上,他们的成果没有超出Berlyne的想法。类似诸如更多的注视个数、更长的注视时长、随时间更富变化的扫视范围等实验结果,仅仅加强证实了Berlyne的观点——对于具有更高美感评价的对象会获得更活跃而动态的眼动行为反馈。事实上,这些结果很难得到合理的美学角度的解释。尽管眼动仪已经成为美学研究的新装备,但到目前为止,就眼动行为是如何与美感反馈产生联系的仍然缺乏深入和令人信服的解释。


\section{眼动与网页美感}
网页美感的研究主要都聚焦在从网页中提取具有美感推测力的特征上,例如复杂度和秩序\citen{Deng2010},低级的图像特征\citen{Zheng}和高级客观设计指标\citen{Ivory}。
Seckler et al. 2015 考察了设计因素诸如结构和色彩是如何与网页的客观美感的不同方面产生联系的\citen{Seckler2015Linking}。Reinecke et al. 2013 引入对网页视觉复杂度和色彩性的计算模型,发现它们对人类的美感偏好具有预测能力\citen{Reinecke}。

眼动仪被广泛用来评估对网页上特定元素的视觉注意度,然而现有的眼动数据的可视化和分析手段与美感相关的用户体验没有任何关联,关于眼动和美感的关系仍有待证实\citen{Santella}。眼动仪能否提供一个更为普遍的度量观者的美感体验的手段?我们需要通过美学的视角借助一些数据挖掘的手段,来从眼动数据中提取出更多的可解释的指标。


\section{网页版式美感计算}

对网页的美学研究,主要集中在两个领域:

传统心理学领域采用激励-反馈实验验证与网页美感可能相关的指标,如视觉注意分布\citen{Djamasbi2011},复杂性\citen{Michailidou2008, Tuch2012Is}等。数学家Birkhoff 1933在他的著作$Aesthetic ~Measure$中提出复杂性的概念,认为美与事物内在的秩序成正比与复杂性成反比。复杂性与吸引视觉注意的图像中的元素的数量有关,而秩序是在图像中呈现的规律性的数量,这一理论中关于视觉注意和复杂性的概念对后续的设计和图像美学研究影响很大。

而计算领域的研究人员以更精确的网页美感分类和预测为目的,采用图像处理手段,挖掘网页美感的视觉特征:Harrington等使用视觉元件左边缘的投影构成直方图\citen{Harrington2004};Zheng提出了一种基于四叉树网页自动分割方法\citen{Zheng},通过变换找出均衡、空白等版式布局特征与主观审美评价的关系;中科院自动化所Wu等人尝试从版式、文本、颜色和纹理、复杂度,四个角度构建网页美感的特征矢量\citen{Wu2011},用于网页美感评价。哈佛大学Reinecke等人基于较大数据量分析,验证了网页中与复杂性和颜色两个方面的一系列特征对美感有一定的推测能力\citen{Reinecke}。

网页的版式评分系统往往被应用在自动化生成式设计系统中。现有的一些版式评分系统采用基于规则的专家模式,如Gaudii平面设计专家系统\citen{Gonzalez2010},利用专家设定的128条模糊逻辑规则作为适应函数,在对海报的版式美感评价中取得了较好的美学效果。但是,现实中的设计存在大量原则之外的特例,手工定义所有版式规则和适用条件是几乎不可能的。因此,基于统计学习的路径更值得期待,2014年多伦多大学O’Donovan开发了一个带有学习能力的版式系统\citen{O2014},采用了一个由122个版式特征变量的能量函数评估自动生成的方案,该函数中各项的权重设置以类似模拟退火的非线性逆优化(NIO)方式从设计师选择的案例中学习,取得了一定的效果。不足的是该函数的本质是案例模仿,而不是更具推广性的审美计算模型,且这些版式特征量的有效性并没有经过大样本量的验证和筛选。总体而言,网页版式评价在数据采集、特征选择和学习方法上还有很大的尝试和发展空间。

\chapter{理论假设}
\label{chap:hypothesis}
我们的设想源自流畅理论\citen{Reber2004, Reber2012}。浏览一个网页的过程本质上是人类的视觉系统处理一张图像的过程。如何来评估这种视觉处理的流畅性呢?我们可以通过视觉处理管道的入口,神经和认知系统的上游——视觉注意,来入手。根本上,视觉注意可以定义为从全部可达的信息中选出一个用于进一步处理的子集的过程。视觉注意持续地通过自底向上(画面视觉重点的驱动)和自定向下(诱人的内容的驱动)的方式在外围视力中选择目标。审美主体的视觉注意受他对审美对象的视觉复杂度(Complexity)和视觉重点分布(即显著性分布,Saliency)两个部分。前者决定了耗费视觉注意资源量的大前提,后者则可以在相同的视觉复杂度下降低或增高视觉处理的代价。

视觉接收通道可以理解成一个具有带宽(视觉信息接收能力)上限的网络信息传输通道。如果流畅理论成立的话,则注意过程将是美感评价的途径。具体而言,一个能够造成正面审美反馈的对象的视觉复杂度应该在一个不过高亦不过低的合理的区间范围内,在此基础上的视觉重点分布应该能够恰当地引导视线,从而提高视觉信息传输的效率。

合理而流畅的视觉注意应满足:
\begin{enumerate}
  \item 在面对多个视觉注意线索时,更少冲突地作出选择.
  \item 在面对局部视觉重点时,更聚焦地趋向一个兴趣点
\end{enumerate}

本研究中,对于审美主体,我们引入信息熵的概念来量化视觉注意行为的混乱度(不流畅性),并探究它是否跟美感评价相关。熵将被应用在眼动数据的以下两个方面:

\begin{enumerate}
  \item 眼动注意的转移序列的熵:反映注意力在多个视觉重点间转移的复杂度和不确定性
  \item 眼动注意的热图熵:反映眼动注意在空间分布上的分散度和混乱度。
\end{enumerate}

对于审美客体,我们引入一些已经被提出或是新的用于度量视觉复杂度和视觉显著性分布的图像特征,并探究由他们搭建的学习模型对美感的推测力。

%# -*- coding: utf-8-unix -*-
%%==================================================
%% conclusion.tex for SJTUThesis
%% Encoding: UTF-8
%%==================================================

\begin{summary}

本文通过一个眼动实证研究,一个图像特征推测力研究和一个网页版式评分系统的搭建,在理论和应用层面多角度地证实了进化论美学和流畅理论关于视觉复杂度和视觉注意力与美感的关系的猜想。眼动实验提出了视觉注意熵的概念,论证了拥有较强美感的审美对象会导致较小的相对视觉注意熵;图像特征提取实验通过对网格复杂度、占空分布、信息密度分布、视觉显著性分布等特征的提取及验证、论证了网页截图图像中关于视觉复杂度的信息与关于视觉重点分布的特征对美感的显著推测能力;最后,对美感评分系统的具体的工程实现和83\%的网页版式好坏的区分正确率,切实给出了机器获得审美的可行性和技术架构。

上述结论表明,从理论层面上,审美体验遵循最小代价最大收益的原则\citen{Hekkert2006}。“美即是可用”\citen{Tractinsky2000}的观点至少对于先天性的审美而言是正确的。

工程上,网页版式评分系统本身还有诸多值得探索和突破的细节。一个全面而强大的美感评分系统应该是由多个评估模型(如版式、色彩、物件识别、语义等)整合而成的,能对审美对象进行全方位评价的系统。

设计自动化亦是美感评分系统的一个重要的应用方向。基于高效的版式生成式系统和优秀的美感评分系统的生成式设计系统的诞生是令人期待的。然而毋庸置疑,对美学的深入理解和研究是获得高效且优雅\footnote{此处“优雅”指结构清晰可解释}的基石。

\end{summary}


\appendix	% 使用英文字母对附录编号,重新定义附录中的公式、图图表编号样式
\renewcommand\theequation{\Alph{chapter}--\arabic{equation}}
\renewcommand\thefigure{\Alph{chapter}--\arabic{figure}}
\renewcommand\thetable{\Alph{chapter}--\arabic{table}}
\renewcommand\thealgorithm{\Alph{chapter}--\arabic{algorithm}}

%% 附录内容,本科学位论文可以用翻译的文献替代。
\include{tex/app_setup}
\include{tex/app_eq}
\include{tex/app_cjk}
\include{tex/app_log}

\backmatter	% 文后无编号部分

%% 参考资料
\printbibliography[heading=bibintoc]

%% 致谢、发表论文、申请专利、参与项目、简历
%% 用于盲审的论文需隐去致谢、发表论文、申请专利、参与的项目
\makeatletter

%%
% "研究生学位论文送盲审印刷格式的统一要求"
% http://www.gs.sjtu.edu.cn/inform/3/2015/20151120_123928_738.htm

% 盲审删去删去致谢页
\ifsjtu@review\relax\else
  %# -*- coding: utf-8-unix -*-
\begin{thanks}

本文的研究,尤其是眼动实验的相关研究,历时较长。实验期间顾振宇教授长期鼓励和指导,令我有动力去探索更多的解释性指标,并最终取得一定成果。论文撰写上,顾振宇教授的反复修改与指导使结果更为深入和有说服力,令我受益匪浅。

另在实验和工程实现方面,感谢媒设2012级研究生娄坚的特征提取平台;感谢电院2017级研究生邱丰对众包实验平台的搭建上的贡献;感谢电院2017级研究生邓瀚铭对美感评分的机器学习卷积神经网络的搭建;感谢媒设2017级研究生杨秀凡对眼动实验中例外页面的设计改进;感谢媒设2015级研究生张杰琳对代码的重构意见;感谢媒设2015级研究生王靖纯对实验的协助。

感谢所有参与眼动实验和网页众包评分实验的约80名同学和师长。

\end{thanks}
 	  %% 致谢
\fi

\ifsjtu@bachelor
  % 学士学位论文要求在最后有一个英文大摘要,单独编页码
  \pagestyle{biglast}
  \include{tex/end_english_abstract}
\else
  % 盲审论文中,发表学术论文及参与科研情况等仅以第几作者注明即可,不要出现作者或他人姓名
  \ifsjtu@review\relax
    %# -*- coding: utf-8-unix -*-

\begin{publications}{99}
    \item\textsc{第一作者}. {中文非核心期刊论文}, 2017.
\end{publications}

    \include{tex/projectsreview}
  \else
    %# -*- coding: utf-8-unix -*-
%%==================================================
%% pub.tex for SJTUThesis
%% Encoding: UTF-8
%%==================================================

\begin{publications}{99}
  \item\textsc{金辰浩}. {基于互联网大数据的设计语义模型}[J]. 工业设计, 2017/10(135): 54-55.
  % \item\textsc{Chen H, Wu B~I, Zhang B}, et al. {Electromagnetic Wave Interactions with a Metamaterial Cloak}[J]. Physical Review Letters, 2007, 99(6):63903.
\end{publications}
	      %% 发表论文
    \include{tex/projects}  %% 参与的项目
  \fi
\fi

% \include{tex/patents}	  %% 申请专利
% \include{tex/resume}	  %% 个人简历

\makeatother

\end{document}
